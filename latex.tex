\documentclass[11pt,a4paper]{article}
\usepackage{graphicx}
\usepackage[a4paper,bottom = 1.5cm,left = 1.5cm,right = 1.5cm,top = 1.5cm]{geometry}
\usepackage[rightcaption]{sidecap}
\usepackage[utf8]{inputenc}
\usepackage[english]{babel}
\usepackage[document]{ragged2e}
\usepackage{titling}
\usepackage{color}
\usepackage{enumitem}
\usepackage{float}
\usepackage{fancyhdr}
\pagestyle{fancy}
\setlength{\parindent}{7em} 
\lhead{190050055-190050077-190050079}
\rhead{Wirless Measurments}
\cfoot{Page \thepage}

\title{\textbf{LAB 1 : Wirless Measurments}}
\author{190050055,190050077,190050079}
\date{18 Jan 2021} 
\cleardoublepage

\begin{document}

\maketitle 


\newpage
{\huge A)}\\

\vspace{0.5cm}
For this lab assignment , two of our team-mates recorded signal strength. And the recorded data is pictorially represented below. 

\begin{SCfigure}[0.5][h]
\caption{Kranthi.png}
\includegraphics[width=0.7\textwidth]{kranthi-1.png}
\end{SCfigure}
\begin{SCfigure}[0.5][h]
\caption{Sonnandh.png}
\includegraphics[width=0.7\textwidth]{sonnandh-1.png}
\end{SCfigure}
For Figure 1, for signal strength ranging from \textcolor{green}{0dm to -85dm}  corresponds to \textbf{green color}, for signal strength ranging from \textcolor{yellow}{-85dm to -105dm} corresponds to \textbf{yellow color}, and for signal strength  \textcolor{red}{$\leq$ -105dm} corresponds to \textbf{red color}. 
\\
\vspace{8pt}
\newpage
{\huge B)}\\

\vspace{0.5cm}
\textbf{\Large{Figure-1}}\\
During the measurement of signal strength for Figure 1, phone was connected to a total of three unique-base-stations (eNodeBIDs), having eNodeBIDs as follows:
\vspace{-8pt}
\begin{itemize}[itemsep = -2 mm]
\item 941076
\item 941063
\item 941105
\end{itemize}
\vspace{1cm}
\textbf{\Large{Figure-2}}\\
 During the measurement of signal strength for Figure 2, phone was connected to a total of three unique-base-stations (eNodeBIDs), having eNodeBIDs as follows:
 \vspace{-8pt}
 \begin{itemize}[itemsep = -2mm]
     \item1568
 \item6185
 \item132086
 \item931
 \item9021
 \item132188
 \end{itemize}
 \vspace{1cm}
 {\huge C)}\\

 \vspace{0.5cm}
\textbf{For Fig 1:}\\
 The reasons for poor connection may be because the phone is far away from the base station, and that region is more crowded when compared to other regions during the walk(Due to more crowded ). The sudden change from red to green is due to change in connection with the base station(i.e phone is now connected to a different base station which is near-by with better signal).
\\
\vspace{1cm}
\textbf{For Fig 2:}\\
Since we can see that the signal was green and yellow for the most of 
the time for Fig 2. This means that the signal strength 
was good but the signal did change from green to yellow ,The reason for this is that If we
notice properly in the map of Fig 2 the signal changed at the main road which generally 
has more traffic  and also has more
buildings when compared to the inner parts of the town and we can see that it is indeed 
more greener i.e, more stronger signal at inner parts of the town. This is due to destructive
interference of the signal waves with possibly other
signal waves or also itself by means of reflection with nearby surroundings and also one of the
reasons is that  Cellular signals have a hard time passing through 
metal and concrete which are essential materials of most buildings and another reason for poor signal could also 
be high Cellular Traffic at that moment of observation.
\vspace{8pt}
\newpage
{\huge D)}\\

\vspace{0.5cm}
 While standing in a fixed position, the signal strength (\textbf{RSRP: Reference signal received power}) varies with time. The image below is plot representing ,how RSRP varies with time at a fixed location.\\
\textbf{For Fig 2 :}\\
\begin{SCfigure}[0.5][h]
    \centering
\caption{RSRP(in dms) vs time(in secs)}
\includegraphics[width=0.7\textwidth]{WhatsApp Image 2021-01-17 at 10.57.02 PM.jpeg}
\end{SCfigure}


\end{document}

