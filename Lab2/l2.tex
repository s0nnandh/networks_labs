\documentclass{article}
\usepackage[a4paper,left=1in,right=1in,bottom=1in]{geometry}
\usepackage[parfill]{parskip}
\usepackage{ragged2e}
\usepackage{amsmath} 
\usepackage{amsthm} 
\usepackage{graphics}
\usepackage{amssymb}
\usepackage{upgreek}
\usepackage{listings}
\usepackage{esint}
\usepackage{floatrow}
\newtheorem{theorem}{Theorem}
\newtheorem{lemma}{Lemma}
\newtheorem{corollary} [theorem] {Corollary}
\newtheorem{proposition}{Proposition}[section]
\theoremstyle{remark}
\newtheorem*{remark}{Remark}
\usepackage[ruled,linesnumbered,vlined]{algorithm2e}
\usepackage{xcolor}
\usepackage{listings}
\usepackage{color}
\definecolor{name}{rgb}{0.5,0.5,0.5}
\usepackage{hyperref}
\usepackage{url}
\usepackage{multirow}
\usepackage{fancyhdr}
\usepackage{graphicx}
\newcommand{\y}[1]{\mathcal{O}(#1)}
\pagestyle{fancy}
\fancyhf{}
%\setlength{\parindent}{7em} 
\lhead{190050055-190050077-190050079}
\rhead{Network Diagnostic Tools}
\cfoot{Page \thepage}
\renewcommand{\footrulewidth}{1pt}
\newcommand{\tbf}[1]{\textbf{#1}}
\setcounter{tocdepth}{2}
\usepackage[utf8]{inputenc}
\begin{document}
\title{Lab2-Network Diagnostic Tools}
\author{190050055-190050077,1900500079}
\date{January, 2021}
\maketitle
%\tableofcontents
\thispagestyle{empty}
\clearpage
\pagenumbering{arabic}



\large{Ipconfig}

\begin{table}[H]
    \centering
    \resizebox{\textwidth}{!}{%
    \begin{tabular}{|c| c| c| c|}
         \hline
         \textbf{Fields} & \textbf{Sonnandh} & \textbf{Bharani} & \textbf{Kranthi}\\
         \hline
         $Interface$& Wifi & Wifi & Wifi \\
         \hline
         $Ipv4$& 192.168.0.100 & 192.168.0.100  & 192.168.0.100 \\
         \hline
         $IPv6$&  fd01::fa2e:2190:4a08:3142 & fd01::fa2e:2190:4a08:3142  & fd01::fa2e:2190:4a08:3142 \\
         \hline
         $MAC$&  6C:6A:77:8E:FF:53 & 6C:6A:77:8E:FF:53   & 6C:6A:77:8E:FF:53 \\
         \hline
         $Queue length$ &  1000 & 1000 & 1000\\
         \hline
         $MTU$ &  1500 & 1500 & 1500\\
         \hline        
          
    \end{tabular}%
}
    \caption{Ipconfig Data on respective local amchines}
    \label{Table 1:}
\end{table}

% \begin{table}[H]
%     \centering
%     \begin{tabular}{|c| c| c|}
%                  \hlinebouygues.iperf.fr
%                  \textbf{Sonnandh} & \textbf{Bharani} & \textbf{Kranthi}\\
%                  \hline
%                  $Interface$& Cabi & uhcueg \\
%                  \hline
%                  $Ipv4$& safc $(-1)^n$ & uhcueg \\
%                  \hline
%                  $IPv6$& euvgy $(-1)^n$ & uhcueg \\
%                  \hline
%                  $MAC$& dvcgavuh  $(-1)^n$ & uhcueg \\
%                  \hline
%                  $MTU$ & idvuyvc & ugduysv\\
%                  \hline        
                  
%             \end{tabular}
%     \caption{Some common time complexities}
%     \label{Table 1:}
% \end{table}
  
\tbf{MTU} stands for \tbf{Maximum Transmission unit} which is measured in \tbf{bytes}.

\large{Traceroute}



\large{Ping}

\begin{table}[H]
    \centering
    \resizebox{\textwidth}{!}{%
    \begin{tabular}{|c| c| c| c|}
         \hline
         \textbf{Domain Name} & \textbf{IP Adress} & \textbf{Location} & \textbf{RTT(in ms)}\\
         \hline
         google.com& 8.8.8.8 & USA & 40.043 \\
         \hline
         google.com& 172.217.163.68 & USA  & 38.075 \\
         \hline
         cloudflare&  1.1.1.1 & Australia  & 46.195 \\
         \hline
         telstra.com.au&  139.130.4.5 & Australia   & 196.374 \\
         \hline
         $Queue length$ &   205.210.42.205 & 1000 & 1000\\
         \hline
         $MTU$ &  1500 & 1500 & 1500\\
         \hline        
          
    \end{tabular}%
}
    \caption{Ping Data}
    \label{Table 1:}
\end{table}

\tbf{\large{Iperf}}

On connecting to the server \tbf{bouygues.iperf.fr} which is located in France: \\
Observed \tbf{TCP} Bandwidth rate (in Mbps) : 2.49  \\
Observed throughput for 1Mb \tbf{UDP} (in Mbps) : 0.13 \\



On connecting to the server \tbf{iperf.biznetnetworks.com} which is located in Indonesia: \\
Observed \tbf{TCP} Bandwidth rate (in Mbps) : 2.23  \\
Observed throughput for 1Mb \tbf{UDP} (in Mbps) : 0.13 \\

We can see that \tbf{TCP} bandwidth rate is much higher than \tbf{UDP} because \tbf{TCP} 
makes better use of the bandwidth.
The reason is because TCP will try and buffer the data and fill a full network segment thus making more efficient use of the available bandwidth.

UDP on the other hand puts the packet on the wire immediately thus congesting the network with lots of small packets.


\end{document}
 