\documentclass{article}
\usepackage[a4paper,left=1in,right=1in,bottom=1in]{geometry}
\usepackage[parfill]{parskip}
\usepackage{ragged2e}
\usepackage{amsmath} 
\usepackage{amsthm} 
\usepackage{graphics}
\usepackage{amssymb}
\usepackage{upgreek}
\usepackage{listings}
\usepackage{esint}
\usepackage{floatrow}
\newtheorem{theorem}{Theorem}
\newtheorem{lemma}{Lemma}
\newtheorem{corollary} [theorem] {Corollary}
\newtheorem{proposition}{Proposition}[section]
\theoremstyle{remark}
\newtheorem*{remark}{Remark}
\usepackage[ruled,linesnumbered,vlined]{algorithm2e}
\usepackage{xcolor}
\usepackage{listings}
\usepackage{color}
\definecolor{name}{rgb}{0.5,0.5,0.5}
\usepackage{hyperref}
\usepackage{url}
\usepackage{multirow}
\usepackage{fancyhdr}
\usepackage{graphicx}
\newcommand{\y}[1]{\mathcal{O}(#1)}
\pagestyle{fancy}
\fancyhf{}
%\setlength{\parindent}{7em} 
\lhead{190050055-190050077-190050079}
\rhead{Network Diagnostic Tools}
\cfoot{Page \thepage}
\renewcommand{\footrulewidth}{1pt}
\newcommand{\tbf}[1]{\textbf{#1}}
\setcounter{tocdepth}{2}
\usepackage[utf8]{inputenc}
\begin{document}
\title{Lab2-Network Diagnostic Tools}
\author{190050055-190050077,1900500079}
\date{January, 2021}
\maketitle
%\tableofcontents
\thispagestyle{empty}
\clearpage
\pagenumbering{arabic}



\large{Ipconfig}

\begin{table}[H]
    \centering
    \resizebox{\textwidth}{!}{%
    \begin{tabular}{|c| c| c| c|}
         \hline
         \textbf{Fields} & \textbf{Sonnandh} & \textbf{Bharani} & \textbf{Kranthi}\\
         \hline
         $Interface$& Wifi & Wifi & Wifi \\
         \hline
         $Ipv4$& 192.168.0.100 & 192.168.0.100  & 192.168.0.100 \\
         \hline
         $IPv6$&  fd01::fa2e:2190:4a08:3142 & fd01::fa2e:2190:4a08:3142  & fd01::fa2e:2190:4a08:3142 \\
         \hline
         $MAC$&  6C:6A:77:8E:FF:53 & 6C:6A:77:8E:FF:53   & 6C:6A:77:8E:FF:53 \\
         \hline
         $Queue length$ &  1000 & 1000 & 1000\\
         \hline
         $MTU$ &  1500 & 1500 & 1500\\
         \hline        
          
    \end{tabular}%
}
    \caption{Ipconfig Data on respective local amchines}
    \label{Table 1:}
\end{table}

% \begin{table}[H]
%     \centering
%     \begin{tabular}{|c| c| c|}
%                  \hlinebouygues.iperf.fr
%                  \textbf{Sonnandh} & \textbf{Bharani} & \textbf{Kranthi}\\
%                  \hline
%                  $Interface$& Cabi & uhcueg \\
%                  \hline
%                  $Ipv4$& safc $(-1)^n$ & uhcueg \\
%                  \hline
%                  $IPv6$& euvgy $(-1)^n$ & uhcueg \\
%                  \hline
%                  $MAC$& dvcgavuh  $(-1)^n$ & uhcueg \\
%                  \hline
%                  $MTU$ & idvuyvc & ugduysv\\
%                  \hline        
                  
%             \end{tabular}
%     \caption{Some common time complexities}
%     \label{Table 1:}
% \end{table}
  
\tbf{MTU} stands for \tbf{Maximum Transmission unit} which is measured in \tbf{bytes}.

\tbf{\large{Traceroute}}
\normalsize

Domain Name : \tbf{google.com} \\

From local machine           : Took 7 hops and reached 142.250.71.36 Ip adress with RTT 46.6ms.\\
From US(Seattle)           : Took 8 hops and reached 172.217.3.196 Ip adress with RTT 24.935ms. \\
From Croatia(Prague)         : Took 7 hops and reached 216.58.201.68 Ip adress with RTT 15.45ms. \\
From Saudi-Arabia(Riyadh)  : Took 13 hops and reached 172.217.169.68 Ip adress with RTT 79.840ms. \\
From Austria(Vienna)       : Took 6 hops and reached 172.217.23.100 Ip adress with RTT 13.25ms. \\
From South Korea(Seoul)      : Took 17 hops and reached 172.217.175.228 Ip adress with RTT 27.455ms.\\  


Domain Name : \tbf{cnn.com} \\

From local machine           : Took 5 hops and reached 151.101.153.67 Ip adress with RTT 47.042ms.\\
From US(Seattle)           : Took 5 hops and reached 151.101.53.67 Ip adress with RTT 22.469ms. \\
From Croatia(Prague)         : Took 8 hops and reached 199.232.17.67 Ip adress with RTT 13.528ms . \\
From Saudi-Arabia(Riyadh)  : Took 6 hops and reached 151.101.1.67 Ip adress with RTT 22.6ms. \\
From Austria(Vienna)       : Took 7 hops and reached 151.101.1.67 Ip adress with RTT 10.838ms. \\
From South Korea(Seoul)      : Took 11 hops and reached 151.101.109.67 Ip adress with RTT 24.44ms.\\ 

Domain Name : \tbf{iitd.ac.in} \\

From local machine           : Took 16 hops and reached 103.27.9.24 Ip adress with RTT 64.468ms.\\
From US(Seattle)           : Took 12 hops and reached 103.27.9.24 Ip adress with RTT 46.6ms. \\
From Croatia(Prague)         : Took 19 hops and reached 103.27.9.24 Ip adress with RTT 46.6ms.  \\
From Saudi-Arabia(Riyadh)  : Took 10 hops and reached 103.27.9.24 Ip adress with RTT 46.6ms. \\
From Austria(Vienna)       : Took 16 hops and reached 103.27.9.24 Ip adress with RTT 46.6ms. \\
From South Korea(Seoul)      : Took 19 hops and reached 103.27.9.24 Ip adress with RTT 46.6ms.\\  



\large{Ping}

\begin{table}[H]
    \centering
    \resizebox{\textwidth}{!}{%
    \begin{tabular}{|c| c| c| c|}
         \hline
         \textbf{Domain Name} & \textbf{IP Adress} & \textbf{Location} & \textbf{RTT(in ms)}\\
         \hline
         google.com& 8.8.8.8 & USA & 40.043 \\
         \hline
         google.com& 172.217.163.68 & USA  & 38.075 \\
         \hline
         cloudflare&  1.1.1.1 & Australia  & 46.195 \\
         \hline
         telstra.com.au&  139.130.4.5 & Australia   & 196.374 \\
         \hline
         $Queue length$ &   205.210.42.205 & 1000 & 1000\\
         \hline
         $MTU$ &  1500 & 1500 & 1500\\
         \hline        
          
    \end{tabular}%
}
    \caption{Ping Data}
    \label{Table 1:}
\end{table}

\tbf{\large{Iperf}}\\

\normalsize

On connecting to the server \tbf{bouygues.iperf.fr} which is located in France: \\
For \tbf{TCP} : \\
Observed \tbf{TCP} Bandwidth rate (in Mbps) : 22.2  \\
For \tbf{UDP} :\\
Observed throughput for 1Mb  (in Mbps) : 1.03 \\
Observed throughput for 2Mb  (in Mbps) : 2.02 \\
Observed throughput for 4Mb  (in Mbps) : 4.04 \\
Observed throughput for 8Mb  (in Mbps) : 8.02 \\
Observed throughput for 16Mb (in Mbps) : 16.1 \\
Observed throughput for 32Mb (in Mbps) : 22.2 \\

Therfore the value of X is 32 Megabits.\\



On connecting to the server \tbf{iperf.biznetnetworks.com} which is located in Indonesia: \\
For \tbf{TCP} : \\
Observed \tbf{TCP} Bandwidth rate (in Mbps) : 19.9  \\
For \tbf{UDP} :\\
Observed throughput for 1Mb  (in Mbps) : 1.08 \\
Observed throughput for 2Mb  (in Mbps) : 2.03 \\
Observed throughput for 4Mb  (in Mbps) : 4.02 \\
Observed throughput for 8Mb  (in Mbps) : 8.00 \\
Observed throughput for 16Mb (in Mbps) : 16.1 \\
Observed throughput for 32Mb (in Mbps) : 19.3 \\

Therfore the value of X is 32 Megabits.\\

We can see that \tbf{TCP} bandwidth rate is much higher than \tbf{UDP} because \tbf{TCP} 
makes better use of the bandwidth.
The reason is because TCP will try and buffer the data and fill a full network segment thus making more efficient use of the available bandwidth.

UDP on the other hand puts the packet on the wire immediately thus congesting the network with lots of small packets.


\end{document}
 